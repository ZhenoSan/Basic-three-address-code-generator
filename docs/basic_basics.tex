\documentclass[a4paper,11pt]{article}

\usepackage{color}
\usepackage{geometry}
\usepackage{hyperref}
\geometry{%
  a4paper,%
  top=2.5cm,%
  bottom=2.5cm,%
  left=2.5cm,%
  right=2.5cm%
}

% beginning documentclass

\begin{document}
 \title{\Huge \color{blue} QBasic Notes}
 \author{Abhay Mittal}
 \date{October 14th, 2014}
 \maketitle
 
 \pagebreak
 
 \section{\color{red}Syntax of Commands}
 \begin{tabular}{| l | l |}
 \hline
  \textbf{Command / Statement} & \textbf{Meaning} \\  \hline
  PRINT & \begin{tabular}{l}This statement is used to print output on the screen \\
  Note: To print more than one item, separate the items by using \\commas (,) and the items are separated by 
  tabs in the output.\\Syntax: PRINT A,B,C \end{tabular} \\ \hline
  
  END & Tells computer that the program has finished \\ \hline
  
  LET & \begin{tabular}{l}Change value of variables \\
  Syntax: LET VarName = VarValue\end{tabular} \\ \hline
  
  
 \end{tabular}
 
 \pagebreak
 
 \section{\color{red}List of Operators}

 \begin{tabular}{| l | l | l |}
  \hline
  \textbf{Operator} & \textbf{Meaning} & \textbf{Priority} \\  \hline
  \^{} & Exponentiation & 1 \\  \hline
  - & Negation. e.g. -23 & 2 \\  \hline
  * & Multiply & 3 \\  \hline
  / & Divide & 3 \\  \hline
  + & Add & 4 \\  \hline
  - & Subtract & 4 \\  \hline
  ' & Comment & NA \\  \hline
  
 \end{tabular}
 \section{\color{red}Variable Naming Rules}
 \begin{itemize}
  \item Var name is up to 40 characters long
  \item Var name begins with alphabet
  \item The rest of the characters must be A-Z, a-z, 0-9, or .
  \item No spaces within the name
  \item Var name must not be in use for something else
  \item Var name isn't case sensitive
  \item The last character of the variable name tells what type of data the memory holds:\\ \\
  \begin{tabular}{| l | l |}
    \hline
    \textbf{Last Character} & \textbf{Data type of variable} \\ \hline
    \% & Integer \\ \hline
    \& & Potentially very big integer \\ \hline
    [No Special symbol] & Floating point number (can have decimal point) \\ \hline
    \# & Potentially very big floating point number \\ \hline
    \$ & String of characters \\ \hline
  \end{tabular}

  \item Consider the following peice of code:\\var\\var=10\\print var\\It will give duplicate definition error because of first line which contains only variable name. That variable becomes unusable.
 \end{itemize}

 
 \section{\color{red}Some important points}
 
 \begin{itemize}
  \item Comma in print produces a tab while printing output
  \item Double quotes ('') are used to encapsulate a string. e.g. ``Hello World''
  \item Division of integers can produce floating quotient
  \item Qbasic can handle floating point powers also (e.g. 10\^{}1.2).
  \item All aritmetic operators are left associative (except exponentiation).
  \item Same priority operators are processed from left to right except exponentiation.
  
 \end{itemize}




\end{document}
