\documentclass[a4paper,11pt]{article}

\usepackage{color}
\usepackage{geometry}
\usepackage{hyperref}
\geometry{%
  a4paper,%
  top=2.5cm,%
  bottom=2.5cm,%
  left=2.5cm,%
  right=2.5cm%
}

% beginning documentclass

\begin{document}
 \title{\Huge \color{blue} QBasic Notes}
 \author{Abhay Mittal}
 \date{October 14th, 2014}
 \maketitle
 
 \pagebreak
 
 \section{\color{red}Syntax of Commands}
 \begin{tabular}{| l | l |}
 \hline
  \textbf{Command / Statement} & \textbf{Meaning} \\  \hline
  PRINT & \begin{tabular}{l}This statement is used to print output on the screen \\
  Note: To print more than one item, separate the items by using \\commas (,) and the items are separated by 
  tabs in the output. \end{tabular} \\ \hline
  END & Tells computer that the program has finished \\ \hline
 \end{tabular}
 
 \pagebreak
 
 \section{\color{red}List of Operators}

 \begin{tabular}{| l | l | l |}
  \hline
  \textbf{Operator} & \textbf{Meaning} & \textbf{Priority} \\  \hline
  \^{} & Exponentiation & 1 \\  \hline
  - & Negation. e.g. -23 & 2 \\  \hline
  * & Multiply & 3 \\  \hline
  / & Divide & 3 \\  \hline
  + & Add & 4 \\  \hline
  - & Subtract & 4 \\  \hline
  ' & Comment & NA \\  \hline
  
 \end{tabular}
 
 \section{\color{red}Some important points}
 
 \begin{itemize}
  \item Double quotes ('') are used to encapsulate a string. e.g. ``Hello World''
  
  \item Division of integers can produce floating quotient
  
  \item Qbasic makes the two meanings of minus signs less confusing by adjusting what we type.
  \begin{enumerate}
   \item When ``-'' means ``negative number'' it is placed right up against the number it is for.
   \item When``-'' means ``subtraction'' it is separated from the numbers to be subtracted by one space on each side.
  \end{enumerate}
  (The adjustment is not done until after the cursor has left the line)
  \\The details of this point can be read \href{http://chortle.ccsu.edu/QBasic/chapter01/bc01_19.html}{here}.
  
  \item Qbasic can handle floating point powers also (e.g. 10\^{}1.2).
  \item All aritmetic operators are left associative (except exponentiation).
  \item Same priority operators are processed from left to right except exponentiation.
  
 \end{itemize}




\end{document}
